\documentclass{beamer}

\usepackage[utf8]{inputenc}
\usepackage[ngerman]{babel}
\usepackage{minted}
\usepackage{hyperref}
\usepackage{color}
\usepackage{tikz}
\usetikzlibrary{decorations.pathreplacing}

\usemintedstyle{emacs}

%\usetheme[progressbar=head,block=fill]{metropolis}

\title{Final Talk LiL4}
\subtitle{Group3 - speedDreams}
\date{July 11, 2017}
\author{Alexander Reisner \and
Alexander Weidinger \and
David Werner}
\institute{Technische Universität München}
\begin{document}
  \renewcommand{\figurename}{\tiny Fig.}
  \maketitle

  \begin{frame}{ToC}
    \tableofcontents
  \end{frame}

  \section{Introduction}
  \subsection{Overview}
  \begin{frame}{Project Overview}
    [ Complete Graphic of the Project ]
  \end{frame}

  \subsection{Trask Description}
  \begin{frame}{Our Task}
    [ List of our points in the specification book ]
  \end{frame}

  \subsection{Sub Projects}
  \begin{frame}{Task allocation}
    \begin{itemize}
      \item Alexander Weidinger
      \begin{itemize}
        \item Extend SpeedDreams 2 (SD2) by a virtual proximity sensor
        \item Build data exchange between SD2 and Simulation Coupler (SimCoupler)
        \item Create data exchange between SimCoupler and QEMU S/A VM
      \end{itemize}
      \item Alexander Reisner
      \begin{itemize}
        \item Introduce the QEMU S/A VM
        \item Exchange data between SimCoupler and QEMU S/A VM
        \item Implement mosquitto client to forward data to the ECUs
      \end{itemize}
      \item David Werner
      \begin{itemize}
        \item Implement an autonomous parking algorithm
        \item Implement mosquitto client to forward calculated control data
      \end{itemize}
    \end{itemize}
  \end{frame}

  \section{Alexander Weidinger}
  \subsection{Proximity Sensor}
  \begin{frame}{Related Work}
    \begin{itemize}
      \item<1-> Research Phase: Find implementations of such sensor for TORCS / SD2
      \item<2-> Adapt the found sensor implementation for usage in SD2 and test it
      \item<3-> Resignation: The sensor isn't appropriate for our use case
      \item<4-> \textbf{Write our own proximity sensor}
    \end{itemize}
  \end{frame}

  \subsection{Implementation Comparison}
  \begin{frame}{Comparison of Implementations}
    \begin{columns}
      \begin{column}{0.5\textwidth}
        \begin{figure}
        \begin{tikzpicture}[scale=0.75]
          % car 1
          \draw (0,0) -- (2,0) -- (2,3) -- (0,3) -- (0,0);
          \fill (1, 1.5) circle [radius=0.05];
          % car 2
          \draw (3,1) -- (5,1) -- (5,4) -- (3,4) -- (3,1);
          \fill (4, 2.5) circle [radius=0.05];
          % line between the two midpoints
          \draw (1, 1.5) -- (4, 2.5);
          % sensor lines
          \foreach \x in {0, 30, ..., 330}
          {
          \draw[red, dashed] (1, 1.5) -- +(\x : 3);
          }
          % distance bracket
          \draw[decoration={brace,mirror,raise=5pt},decorate] (1, 1.5) -- (4, 2.5) node[black,midway,yshift=-0.5cm] {\footnotesize $d_1$};
        \end{tikzpicture}
        \caption{\tiny Proximity sensors implemented by the Simulated Car Racing Championship 2015}
      \end{figure}
      \end{column}
      \begin{column}{0.5\textwidth}
        \begin{figure}
        \begin{tikzpicture}[scale=0.75]
          % car 1
          \draw (0,0) -- (2,0) -- (2,3) -- (0,3) -- (0,0);
          \fill (1, 1.5) circle [radius=0.05];
          % car 2
          \draw (3,1) -- (5,1) -- (5,4) -- (3,4) -- (3,1);
          \fill (4, 2.5) circle [radius=0.05];
          % sensor
          \draw[blue, dashed] (2, 1.5) -- +(0 : 4); % line
          \fill[blue] (2, 1.5) circle [radius=0.05]; % starting point
          % intersections
          \draw[blue] (3, 1.5) circle [radius=0.1];
          \draw[blue]  (5, 1.5) circle [radius=0.1];
          % distance bracket
          \draw[decoration={brace,mirror,raise=5pt},decorate] (2, 1.5) -- (3, 1.5) node[black,midway,yshift=-0.5cm] {\footnotesize $d_2$};
        \end{tikzpicture}
        \caption{\tiny (Laser) proximity sensors implemented by us}
      \end{figure}
      \end{column}
    \end{columns}
  \end{frame}

  \section{Alexander Reisner}

  \section{David Werner}
\end{document}
